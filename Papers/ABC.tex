\documentclass{beamer}
\usepackage{beamerthemeshadow}
\usepackage{beamerthemeDarmstadt}
\usepackage{multicol}
%\usepackage{pgfpages}
%\pgfpagesuselayout{resize to}[letterpaper,border shrink=5mm,portrait]

% \usepackage[english]{babel}
% \usepackage[latin1]{inputenc}
% \usepackage{times}
% \usepackage[T1]{fontenc}
% \usepackage{graphicx,subfigure}
% \usepackage{color}
% \usepackage{amsmath}
% 


% user's definition

% \newtheorem{Thm}{Theorem}[section]
% \newtheorem{Lm}{Lemma}[section]
% \newtheorem{Prop}{Proposition}[section]
% \newtheorem{Cry}{Corollary}[section]
% \newtheorem{Rmk}{Remark}[section]

\newenvironment{changemargin}[2]{%
  \begin{list}{}{%
    \setlength{\topsep}{0pt}%
    \setlength{\leftmargin}{#1}%
    \setlength{\rightmargin}{#2}%
    \setlength{\listparindent}{\parindent}%
    \setlength{\itemindent}{\parindent}%
    \setlength{\parsep}{\parskip}%
  }%
  \item}{\end{list}}

\newcommand{\rad}{\text{rad}}

\title[Learning Your ABC]
{\Large Learning Your ABC}


\author[Eric Errthum]
{Eric Errthum  \\ Winona State University}

\date{\footnotesize January 21, 2009}

\institute[] 
{%Department of Mathematics\\
%University of Maryland, College Park\\
%\vskip 1.0cm
% Quotation
}


% List Outline in the beginning of each section.
% \AtBeginSection[]
% {
%    \begin{frame}
%        \frametitle{Outline}
%        \tableofcontents[currentsection]
%    \end{frame}
% }




%-------------------------------------------------------------------
\begin{document}

\begin{frame}
  \titlepage
\end{frame}



%---------------------------------------------------
\section{Introduction}

\subsection{}
\frame{
{\bf Primes!}
\begin{itemize}
\item<1-> {\bf Definition:} A prime is a positive whole number that cannot be divided evenly by anything except $1$ and itself. 
\end{itemize}
\begin{examples}<2->
$2,\ 3,\ 5,\ 7,\ 11,\ 13,\ \dots$ \uncover<3->{$2^{32582657}-1,\ \dots$} 
\end{examples}
\begin{itemize}
\item<4-> {\bf Awesome Property:} Every rational number can be written as a product of primes to a power.
\end{itemize}
\begin{examples}<5->
\begin{itemize}
\item<5-> $662415793599696251 = 239 \cdot 57301 \cdot 94873\cdot 509833$\vspace{1em}
\item<6-> $\displaystyle\frac{27008742384}{27680640625} = 2^4 \cdot 3^5\cdot 5^{-6}\cdot 11^{-6}\cdot 13\cdot 17^{2}\cdot 43^2$
\end{itemize}
\end{examples}
}

\section{Factorizations}
\subsection{}
\frame{
{\bf Factorizations of Consectutive Numbers}
\vspace{-1em}
\uncover<1>{\begin{eqnarray*}
2 & = & 2 \\
3 & = & 3 \\
4 & = & 2^2\\
5 & = & 5 \\
6 & = & 2 \cdot 3 \\
7 & = & 7\\
8 & = & 2^3\\
9 & = & 3^2 \\
10 & = & 2 \cdot 5\\
11 & = & 11\\
12 & = & 2^2 \cdot 3 \\
{\color{white}\rule{60pt}{10pt}} & & {\color{white}\rule{100pt}{10pt}}\\
\end{eqnarray*}}

\vspace{-250pt}
\uncover<2>{\begin{eqnarray*}
22 & = & 2 \cdot 11 \\
23 & = & 23 \\
24 & = & 2^3  \cdot 3\\
25 & = & 5^2 \\
26 & = & 2 \cdot 23 \\
27 & = & 3^3\\
28 & = & 2^2 \cdot 7\\
29 & = & 29 \\
30 & = & 2  \cdot 3 \cdot 5\\
31 & = & 31\\
32 & = & 2^5\\
{\color{white}\rule{60pt}{10pt}} & & {\color{white}\rule{100pt}{10pt}}\\
\end{eqnarray*}}

\vspace{-250pt}
\uncover<3>{\begin{eqnarray*}
122 & = & 2 \cdot 61\\
123 & = & 3 \cdot 41 \\
124 & = & 2^2 \cdot 31\\
125 & = & 5^3 \\
126 & = & 2 \cdot 3^2 \cdot 7 \\
127 & = & 127\\
128 & = & 2^7\\
129 & = & 3 \cdot 43 \\
130 & = & 2 \cdot 5 13\\
131 & = & 131\\
132 & = & 2^2 \cdot 3 \cdot 11\\
{\color{white}\rule{60pt}{10pt}} & & {\color{white}\rule{100pt}{10pt}}\\
\end{eqnarray*}}

\vspace{-250pt}
\uncover<4>{\begin{eqnarray*}
55122 & = & 2 \cdot 3 \cdot 9187 \\
55123 & = & 199 \cdot 277 \\
55124 & = & 2^2 \cdot 13781\\
55125 & = & 3^2 \cdot 5^4 \cdot 7^2 \\
55126 & = & 2 \cdot 43 \cdot 641 \\
55127 & = & 55127\\
55128 & = & 2^3  \cdot 3 \cdot 2297\\
55129 & = & 29 \cdot 1901 \\
55130 & = & 2 \cdot 5  \cdot 37  \cdot 149\\
55131 & = & 3 \cdot 17 \cdot 23 \cdot 47\\
55132 & = & 2^2 \cdot 7 \cdot 11 \cdot 179 \\
{\color{white}\rule{60pt}{10pt}} & & {\color{white}\rule{100pt}{10pt}}\\
\end{eqnarray*}}

\vspace{-250pt}
\uncover<5>{\begin{eqnarray*}
7796955122 & = & 2 \cdot 11 \cdot 354407051 \\
7796955123 & = & 3^2 \cdot 17 \cdot 50960491 \\
7796955124 & = & 2^2 \cdot 7 \cdot 12527 \cdot 22229\\
7796955125 & = & 5^3 \cdot 62375641 \\
7796955126 & = & 2 \cdot 3 \cdot 1299492521\\
7796955127 & = & 13 \cdot 23 \cdot 3929 \cdot 6637 \\
7796955128 & = & 2^3  \cdot 523 \cdot 1863517\\
7796955129 & = & 3 \cdot 37 \cdot 70242839 \\
7796955130 & = & 2 \cdot 5  \cdot 2777  \cdot 280769\\
7796955131 & = & 7 \cdot 229 \cdot 1487 \cdot 3271\\
7796955132 & = & 2^2 \cdot 3^3 \cdot 2503 \cdot 28843 \\
{\color{white}\rule{60pt}{10pt}} & & {\color{white}\rule{100pt}{10pt}}\\
\end{eqnarray*}}
}

\frame{{\bf Things We Notice}
\begin{itemize}
\item<1-> $a$ and $a+1$ have wildly different factorizations
\item<2-> As $a$ gets really big, the factorizations become mostly "large" primes (like 2503) to a small power (like 1). {\it (Demo)}
\item<3-> Some rare gems are still ``small'' primes to a ``large'' power, e.g. $55125  =  3^2 \cdot 5^4 \cdot 7^2$ and $55962140625= 3^6 \cdot 5^6\cdot 17^3$.
\item<4-> We call these numbers {\bf smooth}.
\end{itemize}
}

\frame{
{\bf Relationships of Prime Factorizations}
\begin{itemize}
\item<1-> If I know the prime factorization of $a$ and $b$, then it's easy to find the prime factorization of $ab$.
\begin{example}<2->
$$
\begin{array}{rclclclclcl}
\uncover<2->{\frac{160000}{1058841} = a &=& 2^8 &\cdot& 3^{-2}& \cdot& 5^{4}& \cdot& 7^{-6}}\uncover<3->{&\cdot &11^0}\vspace{.5em}\\
\uncover<2->{\frac{14235529}{24} = b &=& 2^{-3}& \cdot &3^{-1}}\uncover<3->{&\cdot &5^0} \uncover<2->{&\cdot &7^{6} &\cdot& 11^{2}} \\
\hline\vspace{-.9em}\\
\uncover<2->{\frac{2420000}{27} = ab &=& }\uncover<4->{2^5 &\cdot& 3^{-3}& \cdot& 5^4 }\uncover<4>{&\cdot& 7^0&}\uncover<4->{ \cdot& 11^2}
\end{array}$$
\end{example}
\item<6-> But what about $a+b$?
\end{itemize}
\begin{example}<7-> $$a+b = 2^{-3} \cdot 3^{-2} \cdot 7^{-6} \cdot 40949 \cdot 122698687$$
\end{example}
}

\section{ABCs}
\subsection{}
\frame{{\bf ABC Triples}
\begin{itemize}
\item<1-12> (smooth)*(smooth) = smooth
\item<2-> (smooth)+(smooth) = probably not smooth\\
{\color{white}\rule{100pts}{3pts}} \uncover<3->{BUT SOMETIMES IT IS!}
\end{itemize}
\begin{examples}<4->\vspace{-1em}
\begin{eqnarray*}
17^2 41^2 47^2 + 2^{10} 3^3 5^6 7 &=& 11^4 23^4\\
\uncover<5-11>{{\color{blue}{\color{red}2 \cdot }17^2 41^2 47^2 + 2^{10} 3^3 5^6 7} &{\color{blue}{=}}& {\color{blue}\uncover<6-11>{2 \cdot 353 \cdot 7323377}}}
\end{eqnarray*}\vspace{-4em}

\begin{eqnarray*}
3^{11} 7^4 19^4 23^4 + 2^{10}5^6 11^6 17^6 &=&13^2 67^2 109^2 139^2 157^2 163\\
\uncover<7-11>{{\color{blue}3^{{\color{red}12}} 7^4 19^4 23^4 + 2^{10}5^6 11^6 17^6} &{\color{blue}=}&{\color{blue}\uncover<8-11>{37 \cdot 19749002441821462573}}}
\end{eqnarray*}\vspace{-4em}

\begin{eqnarray*}
2 + 3^{10}109 &=& 23^5\\
\uncover<9-11>{{\color{blue}{\color{red}5 \cdot }2 + {\color{red}5 \cdot }3^{10}109 }&{\color{blue}=}&{\color{blue} \uncover<10-11>{5 \cdot 23^5} \uncover<11>{\text{\ \ \ \ Cheating!}}}}
\end{eqnarray*}
\end{examples}
}

\frame{{\bf Judging ABC triples}
\begin{examples}<1->
\uncover<3->{{\it Better Example: \uncover<5->{(0.96132)}}}\vspace{-.75em}
$$17^2 41^2 47^2 + 2^{10} 3^3 5^6 7 = 11^4 23^4$$
\uncover<2->{{\it Okay Example: \uncover<5->{(0.90013)}}}\vspace{-.75em}
$$3^{11} 7^4 19^4 23^4 + 2^{10}5^6 11^6 17^6 =13^2 67^2 109^2 139^2 157^2 163$$
\uncover<4->{{\it Best Example: \uncover<5->{(1.62991)}}}\vspace{-.75em}
$$2 + 3^{10}109 = 23^5$$
\uncover<6->{{\it Typical Example: (0.36287)}\vspace{-.75em}
$$7 \cdot 5701 + 37 \cdot 1361 = 2^3 \cdot 3 \cdot 3761$$}
\end{examples}
}

\frame{{\bf Measuring ABC triples}
\begin{definition}<1->
The {\bf radical} of a number, $\rad(n)$, is the product of all the primes dividing $n$.

\uncover<2->{{\it Example:} $\rad(2^7 3^4 7^2 101^5) = 2 \cdot 3 \cdot 7 \cdot 101$.}
\end{definition}
\begin{definition}<3->
The {\bf ABC Ratio} of a triple $A+B=C$ is given by
$$\alpha = \alpha(A,B,C) = \frac{\ln(C)}{\ln(\rad(ABC))}.$$
\uncover<4->{{\it Example:} $\alpha(2,3^{10}109,23^5) = \displaystyle\frac{\ln(23^5)}{\ln(2\cdot 3 \cdot 23\cdot 109)} = 1.62991$ }
\end{definition}
}

\frame{{\bf Good ABC Triples}\vspace{-.5em}
\begin{itemize}
\item<1-> {\bf Top three known ABC ratio (verified up to $10^{20}$):} \begin{eqnarray*}
(2,\ 3^{10}109,\ 23^5) & \text{with}  & \alpha = 1.62991\\
(11^2,\ 3^25^67^3,\ 2^{21}3)  & \text{with}  & \alpha = 1.62599\\
(19 \cdot 1307,\ 7 \cdot 29^2 \cdot 31^8,\ 2^83^{22}5^4)  & \text{with}  & \alpha = 1.62349
\end{eqnarray*}
\begin{definition}<2->
A {\bf good ABC triple} is $A+B=C$ where $\alpha(A,B,C) > 1.4$.
\end{definition}
\item<3-> {\bf Largest known good ABC triples:} 
\begin{eqnarray*}
(2^{24}5^5 47^5 181^2,\ 13^{14} 19 \cdot 103 \cdot 571^2 \cdot 4261,\ 7^{28}17\cdot 37^2) \\ \text{with } \alpha = 1.447420 \text{ and 29 digits}\\
(5^9 17^2 23^4 37^2 43 \cdot 4817,\ 3^{14} 11^8 61^2 173^4,\ 2^{52} 19^6 127^2)  \\ \text{with } \alpha = 1.419184 \text{ and 28 digits}
\end{eqnarray*}
\end{itemize}
}

\section{ABC Conjecture}
\subsection{}
\frame{{\bf ABC Conjecture}\vspace{-1em}
\begin{eqnarray*}
\uncover<1->{ \frac{\ln(C)}{\ln(\rad(ABC))} &=&\alpha} \\
\uncover<2->{\ln(C)&=&\alpha \ln(\rad(ABC))} \uncover<3->{= \ln( (\rad(ABC))^\alpha)} \\
\uncover<4->{ C & = & (\rad(ABC))^\alpha}
\end{eqnarray*}\vspace{-1em}
\begin{alertblock}{ABC Conjecture (Oesterle and Masser, 1985)}<5->
For every $\eta >1$, there exists only a finite number of ABC triples such that 
$$C > (\rad(ABC))^\eta$$
i.e. with $\alpha(A,B,C) > \eta$.
\end{alertblock}
}

%\frame{{\bf Examples}
%\begin{changemargin}{-1cm}{-1cm}
%\begin{itemize}
%\item<1-> Consider $(A,B,C)=(17^241^247^2,2^{10}3^35^67,11^423^4)$ with $\alpha = 0.96132$.\\
%\item<2-> Let $\epsilon = 1$ then $11^423^4 \le K_\epsilon(2\cdot 3\cdot 5\cdot 7 \cdot 11 \cdot 17 \cdot 23 \cdot 41 \cdot 47)^2$, so $K_\epsilon \ge 0.00000000135$
%\begin{center}
%\uncover<2->{\begin{tabular}{|c|c|}
%$\epsilon$ & $K_\epsilon \ge$ \\
%\hline
%$0.1$ & $0.28037$\\
%$0.01$ & $1.90285$\\
%$0.001$ & $2.30447$\\
%$0.0001$ & $2.34902$\\
%\vdots & \vdots\\
%\uncover<3->{$0$ & $2.35403$}
%\end{tabular}}
%\end{center}\vspace{1em}
%\item<4-> So take $K_\epsilon = 3$? \uncover<5->{NO, it has to work for EVERY ABC triple. }
%\item<6->For $(A,B,C)=(2,3^{10}109,23^5)$, {\color{blue}$\lim\limits_{\epsilon \rightarrow 0} K_\epsilon = 427.891$}.
%\item<7->For $(A,B,C)=(2^{24}5^5 47^5 181^2,\ 13^{14} 19 \cdot 103 \cdot 571^2 \cdot 4261,\ 7^{28}17\cdot 37^2)$, {\color{red}$\lim\limits_{\epsilon \rightarrow 0} K_\epsilon = 1.96744 \times 10^{12}$}.
%\end{itemize}
%\end{changemargin}
%}

\frame{{\bf Consequences}
\begin{corollary}<1->
There is a largest $\alpha(A,B,C)$. It might be $1.62991$.
\end{corollary}
\begin{corollary}<2->
If the largest $\alpha < 2$, then Fermat's Last Theorem (no integer solutions to $x^n+y^n=z^n$ for $n>2$) is proved.
\end{corollary}
\uncover<3->{{\bf Proof:} Suppose there was a solution, then let $A=x^n$, $B=y^n$, $C=z^n$.\\
Then $\rad(ABC) \le xyz \le z^3$. Applying the conjecture gives $z^n < (\rad(ABC))^2 \le (z^3)^2 = z^6$. Hence $n \le 6$.\\
The cases of $3 \le n \le 6$ were proved in 1825 by Legendre and Dirichlet.}
}

\frame{{\bf More Consequences}
\vspace{-.5em}
\begin{corollary}<1->
If the ABC conjecture is true then the following are also proved:
\vspace{-1em}
\begin{multicols}{2}
\begin{itemize}
\item The generalized Fermat equation
\item Wieferich primes statement
\item The Erdos-Woods conjecture
\item Hall's conjecture
\item The Erdos-Mollin-Walsh conjecture
\item Brocard's Problem
\item Szpiro's conjecture %for elliptic curves
\item Mordell's conjecture
\item Roth's theorem
\item Dressler's conjecture
\item Bounds for the order of the Tate-Shafarevich group
\item Vojta's height conjecture %for curves
\item Greenberg's conjecture
\item The Schinzel-Tijdeman conjecture
\item Lang's conjecture 
\end{itemize}
\end{multicols}
\vspace{-2em}
\begin{center}... and many more!\end{center}
\end{corollary}
}

\frame{{\bf How Close Are We to a Proof?}
\begin{alertblock}{ABC Conjecture (Rephrased)}<2->
Given $\epsilon>0$, there exists a constant $K_\epsilon$ such that for every $A$, $B$, $C$ coprime integers with $A+B=C$, 
$$\log C \le K_\epsilon + (1+\epsilon)\log R$$
where $R=\rad(ABC)$.
\end{alertblock}

\begin{theorem}[Gyory (2007)]<3->
Let $A$, $B$, $C$ be coprime integers with $A+B=C$. Let $t$ be the number of prime factors in $R=\rad(ABC)$. Then 
$$\log C < \frac{2^{10t+22}}{t^{t-4}}R(\log R)^t$$
\end{theorem}
}

\section{PQR}
\subsection{}
\frame{{\bf An Analogy}
\begin{itemize}
\item<1-> Often a strong analogy between integers and polynomials with rational coefficients.
\item<2-> A {\bf prime polynomial} is one that cannot be factorized into smaller polynomials with rational coefficents.
\item<3-> {\it Example:} $x^2+1$ is prime, but $x^2-1 = (x+1)(x-1)$ is not. \uncover<4->{(But $x+1$ and $x-1$ are.)}
\item<5-> Let $\rad(P)$ be the product of all prime polynomials dividing $P$.
\item<6-> {\it Example: }$\rad((x-1)^2(x^2+1)^3) = (x-1)(x^2+1)$.
\item<7-> Let $\deg(P)$ be the degree of the polynomial. Notice that $$\deg(PQ)=\deg(P)+\deg(Q)$$ \uncover<8->{which is just like $\ln(AB)=\ln(A)+\ln(B)$.}
\end{itemize}
}

\frame{{\bf The PQR Theorem}
\begin{itemize}
\item<1-> Replace $A$, $B$, $C$ with polynomials $P$, $Q$, and $R$ and replace $\ln$ with $\deg$.
\end{itemize}
\uncover<2->{\begin{block}{PQR Theorem (Hurwitz, Stothers, Mason)}
Let $P$, $Q$, $R$ be nonconstant relatively-prime polynomials that satisfy $P+Q=R$, then $$\deg(R) < \deg(\rad(PQR)).$$
\end{block}}
}

\frame{{\bf PQR Proof}
\begin{itemize}
\item<1-> First notice that $\displaystyle\frac{F}{\gcd(F,F')} = \rad(F)$.
\item<2> {\it Example:}\vspace{-2em}
\begin{eqnarray*}
F &=&(x-1)^2(x^2+1)^3\\
\text{then } F'&=&2(x-1)(x^2+1)^2(4x^2-3x+1)\\
\text{so } \gcd(F,F') &=& (x-1)(x^2+1)^2\\
\text{and }\frac{F}{\gcd(F,F')} &=& (x-1)(x^2+1) = \rad(F).
\end{eqnarray*}\vspace{-9em}
\item<3-6>$$\begin{array}{rcl}
\uncover<4-6>{{\color{red}P'}}P + \uncover<4->{{\color{red}P'}}Q & = & \uncover<4-6>{{\color{red}P'}}R\\
\uncover<5-6>{{\color{red}-\ \ P}}P' + \uncover<5->{{\color{red}P}}Q' & = & \uncover<5-6>{{\color{red}P}}R'\\
\multicolumn{3}{c}{\uncover<6->{\rule[5pt]{120pt}{.5pt}}}\\
\uncover<6>{P'Q-PQ' &=& P'R-PR'}
\end{array}$$\vspace{-7em}
\item<7-> $P'Q-PQ' = P'R-PR'$.\vspace{.5em}
\item<8->\ \\ \vspace{-3.5em} $$\gcd(R,R')  \left| \frac{P'Q-PQ'}{\uncover<9->{\gcd(P,P')\gcd(Q,Q')}} \uncover<10->{= \frac{P'\rad(Q)}{\gcd(P,P')}-\frac{Q'\rad(P)}{\gcd(Q,Q')}.}\right.$$
\end{itemize}\vspace{-1em}
\noindent\begin{eqnarray*}
\uncover<11->{\deg(\gcd(R,R')) &<& \deg(\rad(Q)) + \deg(\rad(P))}\\
\uncover<13->{\deg\left(\frac{R}{\gcd(R,R')}\right) + }\uncover<12->{\deg(\gcd(R,R')) &<& \!\deg(\rad(PQ))} \uncover<13->{+ \deg(\rad(R))}\\
\uncover<14->{\deg(R) &<& \deg(\rad(PQR))}
\end{eqnarray*}
}

\section{My Work}
\subsection{}
\frame{{\bf What I Did}
\vspace{-1em}
\begin{itemize}
\item<1-> The ABC Conjecture can be generalized to number fields  $\mathbb{Q}(\zeta)$ where $\zeta$ is the root of a rational polynomial. 
\uncover<2->{\begin{example}[Dokchitser]
$\zeta^2-\zeta-3=0 \Rightarrow \zeta =\frac{1+\sqrt{13}}{2} \in \mathbb{Q}(\sqrt{13})$, \vspace{.5em}\\
then $\underbrace{\zeta}_A+\underbrace{(\zeta+1)^{10}(\zeta-1)}_B = \underbrace{2^9(\zeta+1)^5}_C$ \vspace{.5em}\\
This triple has algebraic ABC Ratio of 2.029.
\end{example}}
\item<3-> There are ``interesting'' surfaces in algebraic geometry with ``special'' points that correspond to algebraic numbers.
\item<4-> The corresponding algebraic numbers satisfy $\alpha+\beta=\gamma$ and are usually smooth.
\item<5-> I used some algorithms developed in my thesis to generate 350 of these examples and computed their algebraic ABC ratios.
\end{itemize}
}

\frame{{\bf Results}
\begin{center}
\uncover<2->{\includegraphics[height=1.5in]{ABCGraph.jpg}}
\end{center}
\vspace{-1em}
\begin{itemize}
\item<2-> Points given in order of the degree of the defining polynomial.
\item<3-> None of these ``special'' points correspond to a good ABC example.
\item<4-> Data does follow a trend. Proof? \uncover<5->{No idea how to even begin.}
\item<6-> Failure? \uncover<7->{Well, yes, but no.}
\end{itemize}
}

\frame{{\bf The End?}
\begin{center}
Thanks!
\end{center}
\vspace{3em}

More information:\\
The ABC Conjecture Home Page \\
http://www.math.unicaen.fr/\~{}nitaj/abc.html
}
\end{document}










































