% ========================================================================
% preamble-- INCLUDE, BUT DO NOT CHANGE ANY OF THIS
% ========================================================================

\documentclass{article}
\pagestyle{empty}


% margins, size, formatting

% packages for fancy fonts, symbols, thm/proof environments, etc
\usepackage[left=2cm,top=2cm,right=2cm, bottom=2cm,nohead,nofoot]{geometry}
\usepackage{amssymb,amsmath,amsthm, amscd}
\usepackage{mdframed} \mdfsetup{nobreak=true}

% Theorem Styles
\theoremstyle{problem}
\newtheorem{problem}{Problem}[section]
\newtheorem{theorem}{Theorem}[section]
\newtheorem{lemma}[problem]{Lemma}
\newtheorem{proposition}[problem]{Proposition}
\newtheorem{corollary}[problem]{Corollary}
% Definition Styles
%\theoremstyle{definition}
%\newtheorem{definition}{Definition}[section]
%\newtheorem*{notation}{Definition}
%\newtheorem{example}{Example}[section]
%\newmdtheoremenv[%
%  linewidth=2pt]{framed}{Definition}
%
%\newmdtheoremenv[%
%  linewidth=2pt, linecolor=blue]{done}{Definition}
%

% ========================================================================
% body-- YOU MAY MAKE CHANGES AS NECESSARY BELOW
% ========================================================================

\begin{document}

% Problem/Example Number
{\bf
Problem 4.12
}
% Retype Problem Statement (you can just copy and paste from the lexicon)
Prove that $(1-3x) = \sqrt{5x-1}$ implies $x = 2/9$

\begin{proof}
Suppose $x$ satisfies $(1-3x) = \sqrt{5x-1}$. Squaring both sides yields
\begin{eqnarray*}
 5x-1 &=& (1-3x)^2 \\
5x-1&=& 1-6x+9x^2\\
0 &=&2-11x+9x^2.
\end{eqnarray*}
Using the quadratic formula gives $$x = \frac{11 \pm \sqrt{11^2-4\cdot2\cdot9}}{2\cdot 9} = \frac{11 \pm 7}{18} = 1, \frac{2}{9}.$$
However, the case of $x=1$ must be thrown out since $$(1-3\cdot1) = -2 \neq 2 = \sqrt{5\cdot 1-1}.$$ Hence $x=\frac{2}{9}$.
\end{proof}

\end{document} 
