\documentclass{article}[12pt]
\oddsidemargin=.25in
\evensidemargin=.25in
\topmargin=-.5in
\textwidth=6in
\textheight=9in
\pagestyle{empty}
\usepackage{amsmath,amssymb,amsthm}

\begin{document}
\begin{center}
\bf\Large Writing Guidelines for All Assignments

\normalsize by Wesley Calvert \\
(slightly modified by E. Errthum for the purposes of his course)
\end{center}

Einstein tells us that the whole of science is nothing more than a refinement
of everyday thinking. It is easy in mathematics to bypass the ``thinking'' part,
in favor of calculating, but to do so misses the whole point. One way to make
sure you're thinking (as opposed to just calculating) is to communicate what
you're thinking.
Secondly, while most math graduates will never in their lives after graduation
have to compute the cosets of a normal subgroup, almost all will be required to
communicate technical information to people who don't have the background
to understand it. This is very hard to learn, but it begins with speaking and
writing clearly among ourselves.
For these reasons, good mathematical writing will be required and included
in the grade for every assignment in this course. The remainder of this document
will describe what I mean by ``good mathematical writing.''\\

\section{The Central Thing}
\begin{itemize}
\item What you write {\em must} make sense. Try reading aloud what you have
written. Think about it. Have you said anything ridiculous (it's easier to
do than it sounds, and I routinely root out ridiculous statements from my
writing)? If you replaced every technical term with ``This one thing'' or
``this other thing,'' the people at the campus writing center should be able understand your sentences. All the rest is
elaboration of this one point, that what you write must make sense, and
without it, nothing else matters.
\end{itemize}

\section{Language Issues}
\begin{itemize}
\item At all times, write in complete sentences. Luckily, an equation is a sentence. Firmly fix this in your mind. Of
course, it can also be used as a clause in a larger sentence: ``If $2x^2+3 = 11$,
then $x=\pm2$.'' Notice the correct  connecting' words (``if'' and ''then'') in the previous example. You should be able to read your work aloud verbatim and have it sound connected and flow smoothly.

\item Spelling and grammar should be standard, except where good reason for
a deviation exists.
\item The advice of Strunk and White ({\it Elements of Style, 4ed}) is to be followed
religiously:
\begin{quote}
Vigorous writing is concise. A sentence should contain no unnecessary words, a paragraph no unnecessary sentences, for the same reason that a drawing should have no unnecessary lines
and a machine no unnecessary parts. This requires not that
the writer make all sentences short, or avoid all detail and treat
subjects only in outline, but that every word tell.
\end{quote}
By like token, the same authors wisely counsel, ``Avoid the elaborate, the
pretentious, the coy, and the cute. Do not be tempted by a twenty-dollar
word when there is a ten-center handy, ready, and able.'' While over-used,
under-specific words are generally less desirable, one distinguishing mark I
have seen in the best-educated people I know is that they can, when they
wish, talk more plainly than anyone else.
\item In professional writing (as in this class), the third person is usually preferred over the second, and slightly over the first, but avoiding artificial
language is more important. Also, the active voice is usually preferred
over the passive, but again, good expression is paramount. For instance,
I could not figure out an equally honest and expressive way to write the
last two sentences in the active voice, and to write the present sentence
without the first person would be dishonest.

\item Beware the technical terms ``obvious,'' ``clear,'' ``trivial,'' and their ilk. A
friend of mine once remarked that the mathematical meaning of ``obviously'' was, ``A human could prove it in a finite amount of time without
any specialized knowledge.'' You are almost always better off to either
explain the conclusion or cite where you take it from (Theorem X.y in the
book; Learned in MATH 165; etc.).

\end{itemize}


\section{Audience Issues}
\begin{itemize}
\item You will, except when otherwise instructed, be writing with the class as your audience, not just the instructor.  You may assume that your audience understands the material from prerequisite courses and the present
course to date (although it's not a bad idea to remind the reader of more
obscure points), but you should not assume that the reader is familiar
with material later in the course, material from the book that we haven't read, out-of-class conversations, or any other source not common to all
students in the class.
\end{itemize}

\section{Mathematical Issues}
\begin{itemize}
\item A common abuse of language uses ``='' to transition from one line to
another in a calculation. That symbol should only be used to mean that
two things are identical (e.g. $3 + 2 = 2 + 3$).
\item A proof should clearly describe what it is proving. In an involved argument, it sometimes helps to give an overview of the strategy before starting
the line-by-line proof. All assumptions should be stated. The proof should
have a clear end, at which the reader can tell how the foregoing argument
establishes what was to be proved.
\item Names should call to mind the thing being named, or at least should not
contradict good sense. Re-using the same name within the same or a
related argument is to be avoided.
\item Do not let anything in this guide dissuade you from the frequently invaluable expedients of charts, tables, equations, formulae, pictures, or such.
Use them when they're useful. Just make sure that you explain them, too.
\end{itemize}

\end{document}
