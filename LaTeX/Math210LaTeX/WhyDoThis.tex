\documentclass{article}[12pt]
\oddsidemargin=.25in
\evensidemargin=.25in
\topmargin=-.5in
\textwidth=6in
\textheight=9in
\pagestyle{empty}
\usepackage{amsmath,amssymb,amsthm}

\begin{document}
\begin{center}
\bf\Large Why Are You Being Forced to Do This?
\end{center}

\noindent Just in case I haven't talked about it enough in class, this document lists some of my reasons for requiring you to learn \LaTeX.\\

\noindent The benefits of learning to use \LaTeX\ are many. They include:
\begin{enumerate}
\item Pretty documents. Pride in your work. Attention to detail and style.
\item Typesetting forces you to really focus on the presentation of the homework problem you are typing up, and how to make your work as clear and concise as possible. For each section you write up one or two problems very carefully, making it as perfect as possible. Paying attention to the presentation will also make you pay attention to the logical structure of the argument itself.
\item The primary goal of Math210 is to make a bridge that gets you from the lower-level/computational courses to the upper-level/theoretical courses in the department. Part of the ``mathematical maturity'' process that you are going through concerns learning how to work within a structured logical framework, like dealing with abstract definitions and logical arguments. \LaTeX\ is sort of like a computer version of that, where your notation matters and structure is important. It is a parallel topic that teaches the same skills as the mathematics we are learning.
\item Experience with a sort of mini-programming language, or at least experience with compiling, staying within strict syntax parameters, being able to deal with non-WYSIWYG situations, etc. This could be helpful to you later in your education or your career when you have to write computer programs and/or simple HTML webpages.
\item If you're serious about mathematics and continue in the subject then at some point you WILL have to use \LaTeX. Every mathematics graduate student in the country uses it, and if you go to graduate school you will write your thesis in \LaTeX. Even in industry you will have occasion to make respectable, typed-up mathematical documents. (Last semester when the Math club did its grad school visit, the number one thing the graduate students said was that they wished they'd seen \LaTeX\  before grad school.)
\item It's another skill that you can list on your r{\'e}sum{\'e} that may set you apart from other job applicants.
\item When was the last time you handed in a chicken-scratch handwritten paper in a writing class?
\item Word's Equation Editor is \underline{so} 1993. And it is ugly.
\item \LaTeX\ is cool. So now YOU are cool.
Because you can write stuff up in \LaTeX, you rock.
\end{enumerate}
\end{document}
