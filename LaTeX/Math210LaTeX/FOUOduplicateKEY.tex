\documentclass{article}


% ========================================================================
% preamble 
% ========================================================================

% margins, size, formatting
\oddsidemargin=.25in
\evensidemargin=.25in
\topmargin=-.5in
\textwidth=6in
\textheight=9in
\pagestyle{myheadings}

% packages for fancy fonts, symbols, thm/proof environments, etc
\usepackage{amsmath,amssymb,amsthm}

% Dr. E's special format for HW problems
\newsavebox{\probinput}
\newenvironment{problem}[1]%
  {\vspace{\baselineskip}%
   \ifnum #1>9 \else \hspace{.78ex} \fi {\large \bf#1.$\;$}%
   \begin{lrbox}{\probinput}%
   \begin{minipage}[t]{.9175\textwidth}}%
  {\end{minipage}%
   \end{lrbox}%
   \usebox{\probinput}}
\newsavebox{\subinput}
\newenvironment{subprob}[1]%
  {\vspace{.5\baselineskip}%
   {\bf #1)$\;$}%
   \begin{lrbox}{\subinput}%
   \begin{minipage}[t]{.965\textwidth}}%
  {\end{minipage}%
   \end{lrbox}%
   \usebox{\subinput}}


% ========================================================================
% body 
% ========================================================================

\begin{document}

% header =================================================================
\markright{\large \bf  Student Name \hfill MATH210 HW 3.14 \hfill}
\vspace{\baselineskip}

% problem  ================================================================
\begin{problem}{2} 
Your first LaTeX assignment is to use LaTeX to produce a document that replicates this one
as exactly as possible, with just two differences: First, replace the
name above with your own. Second, make the following letter substitutions so that I know
that you did not just photocopy this document: in Problems 5 and 8, change each $m$ to $n$;
in Problem 13, change each $c$ to $b$. Your grade on this assignment will be based on how much
your paper looks exactly like this one (including these instructions). 

Note: In a regular assignment, for questions with a short answer, you may just  respond in a complete sentence (like in 3 below). For questions asking for a proof, restate the assumptions and the statement that you are trying to prove (but you can leave out definitions). For questions where you grade a proof, you can simply give your grade and explanation (as in 21).
\end{problem}

\begin{problem}{3} 
\begin{subprob}{a} 
$A \cap B = \{1,4,5\}$.
\end{subprob}
\begin{subprob}{d} 
False, because $e>2$.
\end{subprob}
\end{problem}

\begin{problem}{5}
Prove that every integer that is divisible by 6 is even.

\begin{proof}
Suppose $n \in \mathbb{Z}$. Then there is some $k \in \mathbb{Z}$ such that $n = 6k$.
Therefore $n = 2(3k)$, and since $3k$ is also in $ \mathbb{Z}$, this means that $n$ is divisible by 2 and
therefore that $n$ is even.
\end{proof}
\end{problem}

\begin{problem}{8}
%Define $A = \{m \in \mathbb{Z} \mid m^3 - m^2 - 6m = 0\}$. 
Prove that if $n \in A$ then $n= -2$, $0$, or $3$.

\begin{proof}
Note that
\begin{align*}
n^3 - n^2 - 6n &= n(n^2 - n- 6) & \text{(factor out an $n$)}\\
&= n(n + 2)(n- 3). & \text{(factor the quadratic)}
\end{align*}
Therefore since $n \in A  = \{n \in \mathbb{Z} \mid n^3 - n^2 - 6n = 0\}$ then $n(n + 2)(n - 3) = 0$. Thus $n$ must be equal to one of
$-2$, $0$, or $3$.
\end{proof}
\end{problem}

\begin{problem}{13}
Prove that if $a$, $b \in \mathbb{R}$ with $a \le b$ then $[b,\infty) \subseteq [a,\infty)$.

\begin{proof}
Suppose $a \le b$ in $\mathbb{R}$. For all $x \in \mathbb{R}$,
\begin{align*}
x \in [b, \infty) & \Rightarrow x \ge b \\
& \Rightarrow x \ge b \ge a & \text{($b \ge a$ by hypothesis)}\\
& \Rightarrow x \ge a & \text{(transitivity)}\\
& \Rightarrow x \in [a,\infty).
\end{align*}
Therefore we have $[b,\infty) \subseteq [a,\infty)$.
\end{proof}
\end{problem}

\begin{problem}{21} Proofs to grade

\begin{subprob}{h}
Grade: C. This only shows that $A \subseteq B$. It also needs to show that $B \subseteq A$ to establish that $A = B$. Also they should state ``because $f^{-1}(x)$ is onto'' after the third step.
\end{subprob}
\end{problem}

\end{document} 
