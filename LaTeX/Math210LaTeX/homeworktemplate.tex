\documentclass{article}


% ========================================================================
% preamble 
% ========================================================================

% margins, size, formatting
\oddsidemargin=.25in
\evensidemargin=.25in
\topmargin=-.5in
\textwidth=6in
\textheight=9in
\pagestyle{myheadings}

% packages for fancy fonts, symbols, thm/proof environments, etc
\usepackage{amsmath,amssymb,amsthm}
\usepackage{mathrsfs}

% Dr. E's special format for HW problems
\newsavebox{\probinput}
\newenvironment{problem}[1]%
  {\vspace{\baselineskip}%
   \ifnum #1>9 \else \hspace{.78ex} \fi {\large \bf#1.$\;$}%
   \begin{lrbox}{\probinput}%
   \begin{minipage}[t]{.9175\textwidth}}%
  {\end{minipage}%
   \end{lrbox}%
   \usebox{\probinput}}
\newsavebox{\subinput}
\newenvironment{subprob}[1]%
  {\vspace{.5\baselineskip}%
   {\bf #1)$\;$}%
   \begin{lrbox}{\subinput}%
   \begin{minipage}[t]{.965\textwidth}}%
  {\end{minipage}%
   \end{lrbox}%
   \usebox{\subinput}}


% ========================================================================
% body 
% ========================================================================

\begin{document}

% header =================================================================
\markright{\large \bf  Your Name \hfill MATH210 HW 0.0 \hfill}
\vspace{\baselineskip}
% ================================================================

% A problem starts here.
\begin{problem}{6} 
Obviously you should put in the correct problem number. 

\begin{subprob}{c}
This is what you would use if you had a subproblem to do. 
\end{subprob}

\end{problem}
% A problem ends here.

% A problem starts here.
\begin{problem}{8} 
The main question of the problem.

\begin{proof}
If your answer is a proof, this is what you use after restating the question. (You can use this within a subproblem as well.)
\end{proof}

\end{problem}
% A problem ends here.


\end{document} 
