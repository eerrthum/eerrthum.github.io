\documentclass{beamer}
\usepackage{beamerthemeshadow}
\usepackage{beamerthemeDarmstadt}
\usepackage{multicol}
%\usepackage{pgfpages}
%\pgfpagesuselayout{resize to}[letterpaper,border shrink=5mm,portrait]

% \usepackage[english]{babel}
% \usepackage[latin1]{inputenc}
% \usepackage{times}
% \usepackage[T1]{fontenc}
% \usepackage{graphicx,subfigure}
% \usepackage{color}
% \usepackage{amsmath}
% 


% user's definition

\newenvironment{changemargin}[2]{%
  \begin{list}{}{%
    \setlength{\topsep}{0pt}%
    \setlength{\leftmargin}{#1}%
    \setlength{\rightmargin}{#2}%
    \setlength{\listparindent}{\parindent}%
    \setlength{\itemindent}{\parindent}%
    \setlength{\parsep}{\parskip}%
  }%
  \item}{\end{list}}

\title[Your Title Here]
{\Large Your Title Here again}


\author{Your name, or any other subtitle on title slide}

\date{June 31, 1999}

\institute[WSU] {Winona State University}



%-------------------------------------------------------------------
\begin{document}

\begin{frame}
  \titlepage 
\end{frame}



%---------------------------------------------------
\section{Intro}
\subsection{}
\frame{{Welcome to Beamer}
This is beamer: the LaTeX package to make {\color{red} awesome} presentations.\\

Notice how everything in inside of a frame; these define the ``slides''.\\

The most common compiling problem is forgetting to close the frame.

}
\section{Boxes}
\subsection{Boxes and Things}
\frame{{Boxes}
You can make the following boxes:
\begin{theorem}[Theorem Title]
You can omit the theorem title if need be.
\end{theorem}
\begin{example}
Here, I omitted the example title, but I could've included it if I had wanted to.
\end{example}
\begin{alertblock}{Advice}
Throughout your talk, be consistent in your use of color for a box, i.e. Blue only for theorems, Green only for examples, and Red for really important stuff (like Theorems you came up with/proved)
\end{alertblock}

}

\frame{{Beamer working for you}
\begin{definition}
I would've really liked to put this box on the previous slide, but Beamer knows when there's too much on a slide and it doesn't shrink stuff just to fit. If Beamer says it's too much for a slide, it's probably right.
\end{definition}
}

\frame{{Itemized Lists}
\begin{itemize}
\item Notice how horrible the first slide (after the title slide) looks.
\item It's often nicer looking to use itemized lists
\item Don't you think?
\end{itemize}
}

\frame{{Pictures and Graphics}
You can also include pictures:
\begin{center}
\includegraphics[width=2in]{Example.jpg}
\end{center}
(centering optional)\\
\includegraphics[width=2in]{Example.jpg}

Just make sure the graphic file is in the same folder as the .tex file.
}

\section{Advanced Moves}
\subsection{}
\frame{{}
\begin{itemize}
\item Frames don't have to have a title
\item Neither do subsections (I rarely name my subsections, most time leaving them blank like here).
\item If you do use sections, you often have to compile twice. On the first compile a .log file is created so that on the second compile it can make the correct links. It's really bad when you make a change to your talk 5 minutes before you go on and forget to compile twice.
\end{itemize}

}

\frame{{Uncovering}
\begin{itemize}
\item<1->One of the best things Beamer can do, but {\color{green}one of the last things you should implement} is
\item<2-> uncovering.
\item<3-> See source code how to do this.
\item<3-4> It's easiest in an itemized list. You put $<$n-m$>$ after the $\backslash$item. Then that bullet will be shown in steps n through m. (Leaving m blank means until the end, n blank from the beginning).
\item<4-> You can also uncover \uncover<5->{one} \uncover<6->{word} \uncover<7->{at} \uncover<8->{a} \uncover<9->{time.}
\item<10-> This also works inside math equations: $\displaystyle\int e^x\ dx=\uncover<11->{e^x}\uncover<12->{+C}$.
\end{itemize}

}
\end{document}
