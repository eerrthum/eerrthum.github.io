\documentclass{beamer}
\usepackage{beamerthemeshadow}
\usepackage{beamerthemeDarmstadt}
%\usepackage{pgfpages}
%\pgfpagesuselayout{resize to}[letterpaper,border shrink=5mm,portrait]

% \usepackage[english]{babel}
% \usepackage[latin1]{inputenc}
% \usepackage{times}
% \usepackage[T1]{fontenc}
% \usepackage{graphicx,subfigure}
% \usepackage{color}
% \usepackage{amsmath}
% 


% user's definition

% \newtheorem{Thm}{Theorem}[section]
% \newtheorem{Lm}{Lemma}[section]
% \newtheorem{Prop}{Proposition}[section]
% \newtheorem{Cry}{Corollary}[section]
% \newtheorem{Rmk}{Remark}[section]

\newcommand{\bs}{\backslash}
\newcommand{\Z}{\mathbb{Z}}
\newcommand{\Q}{\mathbb{Q}}
\newcommand{\ds}{\frac{\partial}{\partjial s}}
\newcommand{\Af}{\mathbb{A}_f}
\newcommand{\A}{\mathbb{A}}
\newcommand{\C}{\mathbb{C}}
\newcommand{\h}{\mathfrak{h}}
\newcommand{\R}{\mathbb{R}}
\newcommand{\F}{\mathbb{F}}
\newcommand{\K}{{\cal K}}
\newcommand{\vpi}{v}
\newcommand{\RK}{{\cal R}}
\newcommand{\D}{\mathfrak{D}}
\newcommand{\Pro}{\mathbb{P}}
\newcommand{\Po}{\tau}
\newcommand{\SL}{\text{SL}_2}
\newcommand{\GL}{\text{GL}_2}
\newcommand{\PGL}{\text{PGL}_2}
\newcommand{\M}{\text{M}_2}
\newcommand{\tr}{\text{tr}}
\newcommand{\Oh}{\mathcal{O}}
\newcommand{\X}{\mathcal{X}}
\newcommand{\Xs}{\mathcal{X}^*}
\newcommand{\no}{\text{norm}}
\newcommand{\No}{\text{N}}
\newcommand{\Div}{\text{div}}
\font\cute=cmitt10 at 12pt
\font\smallcute=cmitt10 at 9pt
\newcommand{\kay}{{\text{\cute k}}}
\newcommand{\smallkay}{{\text{\smallcute k}}}
\newcommand{\E}{\mathcal{E}}
\newcommand{\q}{{\bf q}}
\newcommand{\Ord}{\text{Ord}}
\newcommand{\ord}{\text{ord}}
\newcommand{\Chi}{\mathcal{X}}
\newcommand{\Hom}{\text{Hom}}
\newcommand{\cond}{\text{cond}}
\newcommand{\alp}{\mathfrak{a}}
\newcommand{\I}{\mathfrak{i}}
\newcommand{\MSL}{\widetilde{\text{SL}_2}}
\newcommand{\sign}{\text{sign}}
\newcommand{\expo}{{\bf e}}
\newcommand{\Weil}{\rho_L}
\newcommand{\Weilb}{\overline{\rho}_{\Lambda_L}}
\newcommand{\Go}{\Gamma_0}
\newcommand{\MGo}{\widetilde{\Gamma_0}}
\newcommand{\ep}{\epsilon}
\newcommand{\lam}{\mathfrak{l}}

\title[Finding Minimal Polynomials with a Norm Calculator]
{\Large Finding Minimal Polynomials with a Norm Calculator}


\author[Eric Errthum]
{Eric Errthum  \\ Winona State University \\ eerrthum@winona.edu}

\date{\footnotesize October 18, 2008}

\institute[] 
{%Department of Mathematics\\
%University of Maryland, College Park\\
%\vskip 1.0cm
% Quotation
}


% List Outline in the beginning of each section.
% \AtBeginSection[]
% {
%    \begin{frame}
%        \frametitle{Outline}
%        \tableofcontents[currentsection]
%    \end{frame}
% }




%-------------------------------------------------------------------
\begin{document}

\begin{frame}
  \titlepage
\end{frame}



%---------------------------------------------------
\section{The Problem}

\subsection{}
\frame{{\bf Algebraic Review}
\begin{itemize}
\item<1-> An {\color{red}algebraic number}, $\zeta$, is a root of an irreducible monic polynomial with rational coefficients.
\begin{examples}<2->\vspace{-1em}
\begin{eqnarray*}
\zeta=i & \Rightarrow & x^2+1=0\\
\zeta=\frac{4}{\sqrt{^3\!\!\!\sqrt{9}+7\ ^3\!\!\!\sqrt{3}}} & \Rightarrow & x^6 + \frac{504}{519} x^4 - \frac{2048}{519} 
\end{eqnarray*}
\end{examples}

\item<3-> This polynomial, $M_\zeta(x)$, is called the {\color{red}minimal polynomial} and $$M_\zeta(x) = \prod_i(x-\sigma_i(\zeta))$$ where $\sigma_i \in \text{Gal}(\Q(\zeta)/\Q)$.

\item<4-> The (absolute) {\color{red}norm of $\zeta$}, $\no(\zeta)= \left|\prod \sigma_i(\zeta)\right|$, is the absolute value of the constant term of $M_\zeta(x)$.

\end{itemize}
}

\frame{{\bf Problem Statement}
\uncover<1->{{\large Given}
\begin{itemize}
\item<1-> A collection of unknown algbraic numbers $\{\zeta_1, \zeta_2, \dots\}$ whose minimal polynomials have known degrees $\{d_1, d_2, \dots\}$.
\item<2-> An algorithm that can compute the norm of an unknown algebraic number (norm calculator).
\end{itemize}}
\uncover<3->{{\large Find}
\begin{itemize}
\item The minimal polynomials for (at least some of) the $\zeta_k$'s?
\end{itemize}}
}

\section{The Motivation}

\subsection{}
\frame{{\bf Shimura Curves}
\begin{itemize}
\item<1-> The Shimura curve $S_6$ is a Riemannian surface of genus zero.
\item<2-> An isomorphism $J: S_6 \stackrel{\sim}{\rightarrow} \Pro^1$ exists.
\item<3-> It can be uniquely specified by choosing the three points that map to $0$, $1$, and $\infty$.
\item<4-> Due to the properties of Shimura curves, no formula exists for such a map.
\end{itemize}
}

\frame{{\bf CM Points on the Shimura Curve}
\begin{itemize}
\item<1-> There is a collection of ``special'' points $\{s_k\} \subset S_6$ called complex multiplication (CM) points.
\item<2-> Three of these, $s_3$, $s_4$, and $s_{24}$, are {\it really} special, so specify \\
$J$ by $$(s_3,s_4,s_{24}) \stackrel{J}{\rightarrow} (0,1,\infty).$$
\item<3-> Then $J$ maps all CM points to algebraic numbers.
\item<4-> There exists an algorithm that calculates $\no(J(s_k))$ for $s_k$ a CM point. (Errthum, 2007)
\item<5-> Using genus theory of groups, we can calculate $d_k$, the degree of $M_{J(s_k)}(x)$.
\end{itemize}

\uncover<6->{{\color{blue}{\bf Back to the problem:}} Let $\{\zeta_k\} = \{J(s_k)\}$.}

}

\section{The Solution}

\subsection{}
\frame{{\bf First Trick}
\begin{itemize}
\item<1-> For a finite number of indices $r$, $M_{J(s_r)}(x)$ has degree 1, i.e. $\zeta_r$ is a rational number.
\item<2-> \ {\color{red}\vspace{-1em}$$\no(\zeta_r)=\pm \zeta_r.$$}
\item<3-> \vspace{-1em} Use a new $J_1$ by taking $(s_3,s_4,s_{24}) \rightarrow (1,0,\infty)$ instead of $(0,1,\infty)$. $$J_1(s)=1-J(s).$$
\begin{example}<4->\vspace{-1em}
\begin{eqnarray*}
\no(J(s_r)) = 4/5 \text{ and }  \no(1-J(s_r))  =  9/5 \\
\Rightarrow \zeta_r = J(s_r) = -4/5
\end{eqnarray*}
\end{example}
\item<5-> Can calculate all rational $\zeta_r$ this way.
\end{itemize}
}

\frame{{\bf Second Trick}
\begin{itemize}
\item<1-> Choose $\zeta_k$ with $d_k$ small.
\item<2-> For $d_k+1$ choices of $r$, specify $J_r(s) = \zeta_r-J(s)$ by $$(s_3,s_r,s_{24}) \stackrel{J_r}{\rightarrow} (\zeta_r,0,\infty).$$
\item<3-> \ \vspace{-2em} \begin{eqnarray*}
\uncover<3->{\no(\zeta_r-J(s_k)) &=& \left| \prod_i\sigma_i(\zeta_r-J(s_k)) \right|\\}
\uncover<4->{&=& \left| \prod_i\sigma_i(\zeta_r-\zeta_k) \right|\\}
\uncover<5->{&=& \left| \prod_i\left( \zeta_r-\sigma_i(\zeta_k)\right)\right|\\}
\uncover<6->{&=& \left| M_{\zeta_k}(\zeta_r) \right|}
\end{eqnarray*}
\end{itemize}
}

\frame{{\bf Brute Force}
\begin{itemize}
\item<1-> So we know $d_k+1$ points on the curve $y=\left| M_{\zeta_k}(x) \right|$.
\item<2-> If it wasn't for the absolute value, we could use a standard polynomial fit and be done. 
\item<3-> Go through the $2^{d_k}$ combinations of minus signs on the values until you find a monic polynomial.\\ (There's only ever one. Proof?)
\item<4-> If there are $R$ rational $\zeta_r$, then we can use this method to find the minimal polynomial of any $\zeta_k$ with $d_k \le R-2$.
\end{itemize}
}

\frame{{\bf Example}
\begin{columns}
\begin{column}{6cm}
\begin{itemize}
\item<1-> $\no(\zeta) = \frac{10}{17}$ and $d=3$.
\item<2-> We use the four CM points that map to $0$, $1$, $\frac{-4}{5}$, and $\frac{2}{3}$ to find the data points:\\
\uncover<3->{\begin{tabular}{c}
$\left(0, \no(\zeta) \right)=\left(0, \frac{10}{17} \right)$\vspace{.5em}\\
$\left(1, \no\left(1-\zeta\right) \right)=\left(1,\frac{25}{102} \right)$\vspace{.5em}\\
$\left(\frac{-4}{5}, \no\left(\frac{-4}{5}-\zeta\right) \right)=\left(\frac{-4}{5}, \frac{5246}{6375} \right)$\vspace{.5em}\\
$\left(\frac{2}{3}, \no\left(\frac{2}{3}-\zeta\right) \right)=\left(\frac{2}{3}, \frac{104}{459} \right)$\vspace{.5em}\\
\end{tabular}}
\end{itemize}
\end{column}
\begin{column}{6.5cm}
\begin{itemize}
\item<4-> Possible minimal polynomials:
{\tiny \begin{eqnarray*}
y&=&\frac{10}{17}-\frac{3316}{5049} x-\frac{1141}{10098} x^2+\frac{718 }{1683}x^3\\
y&=&\frac{10}{17}-\frac{124}{561} x-\frac{83}{374} x^2-\frac{73}{187} x^3\\
y&=&\frac{10}{17}-\frac{812}{459} x-\frac{359}{918} x^2+\frac{278}{153} x^3\\
{\color{red}y}&{\color{red}=}&{\color{red}\frac{10}{17}-\frac{4}{3} x-\frac{1}{2}x^2+x^3}\\
y&=&\frac{10}{17}-\frac{7}{51} x-\frac{24}{17} x^2+\frac{41}{34} x^3\\
y&=&\frac{10}{17}+\frac{137}{459} x-\frac{698}{459} x^2+\frac{7}{18} x^3\\
y&=&\frac{10}{17}-\frac{701}{561} x-\frac{316}{187} x^2+\frac{971}{374} x^3\\
y&=&\frac{10}{17}-\frac{4109}{5049} x-\frac{9082}{5049} x^2+\frac{5989}{3366} x^3
\end{eqnarray*}}
\end{itemize}
\end{column}
\end{columns}\vspace{-.5em}

\uncover<5->{So the minimal polynomial of $\zeta$ is $M_\zeta(x) = \frac{10}{17}-\frac{4}{3} x-\frac{1}{2}x^2+x^3$.}

}

\section*{End}
\frame{{\bf Thanks}

\begin{center}
{\large Questions?}\vspace{10em}

eerrthum@winona.edu
\end{center}
}

\end{document}










































